\documentclass{article}
\title{An Introduction to \LaTeX - A document preparation system}
\author{}
\date{}

\begin{document}
\maketitle

\section{Introduction}
\TeX \ is a typesetting computer program created by Donald Knuth (famous computer scientist). This is quite different from the WYSIWYG2 \footnote{What You See Is What You Get} approach that most modern word processors, such as Microsoft Word take.

\subsection{Motivation}
Knuth was hoping to reverse the \emph{crappy} typographical quality that was prominent in digital printing.
I suppose he was not the only one with that motivation. 
Somebody at {\bf{Apple}} had a similar obsession - Can you guess who?

\subsubsection{Interesting facts}
Donald Knuth on Pronunciation:
\newline
{\it{"I do not get angry when people pronounce TEX in their
favorite way . . . and in Germany many use a soft ch because the X follows the vowel e, not
the harder ch that follows the vowel a. In Russia, ‘tex’ is a very common word, pronounced
‘tyekh’. But I believe the most proper pronunciation is heard in Greece, where you have the
harsher ch of ach and Loch."}}
\par
\LaTeX is pronounced “Lay-tech” or “Lah-tech.”
\noindent
The version number of Tex is converging to $\pi$ and is now at 3.141592653.

\subsection{Why use this system?}
The best thing to do when such a discussion starts is to disappear, since such discussions often get out of hand.
Listed below are some points one could use.
\begin{enumerate}
	\item Because I said so \dots
	\item Writing mathematical formulae
	\item Avoid \emph{tinkering} with layout
	\item Efficiently produce different classes of documents i.e.
	\begin{enumerate}
		\item Article
		\item Thesis
		\item Report
		\item Book
	\end{enumerate}
	\item Add on packages to customise just about everything
\end{enumerate}
Here are some points which people could throw back at you
\begin{itemize}
	\item I am rich \dots
	\item Finding commands
	\item Remembering commands
	\item Dealing with \emph{compiling} issues.
\end{itemize}

\par
Before moving on let's also talk about: 
\begin{description}
	\item [MM] Margin manipulation
	%\usepackage[top=2cm,bottom=2cm,left=2cm,right=2cm]{geometry}
	\item [FM] {\Huge{Font manipulation}}
	\item [BOX] Keep it together man! : mbox, fbox
\end{description}

\subsection{Pragmatism}
Two extreme approaches:
\begin{description}
	\item[First Approach] \hfill \\
	Let \LaTeX take care of \emph{everything}. Simply upload the content from your mind onto the document.
	\item[Second Approach] \hfill \\
	Customise just about everything to do with design and layout. Right down to the font and margins specifications of the footnotes.
\end{description}

\newpage

\section{Equations}

\section{Tables}

\section{Graphics}

\section{Referencing}

\section{Presentation Slides}


\end{document}